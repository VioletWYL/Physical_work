\documentclass[lang=cn]{elegantpaper}

\usepackage{amsmath}


\title{第八章大物作业}
\author{王俞力 \LaTeX}
\institute{学号:2010400736}
\date{\today}

\begin{document}

\maketitle

\section{8.2一容器内储有氧气,其压强为\(1.01\times 10^5Pa\),
温度为\(27.0^{\circ}C\)求:(1)气体分子的数密度;(2)氧气的分子质量;(3)分子的平均速率;(4)分子的平均 平动动能。}

解:
(1)\[
  \begin{align}
    p &= nKT\\
    n &= \frac{p}{KT} \\
    &= \frac{1.01 \times 10^5 Pa}{1.38 \times 10^{-23} J / K \times 300k} \\
    &= 2.44 \times 10^{25} m^{3}
  \end{align}
    \]
(2)
\[
  \begin{align}
    m &= \frac{M}{N_A}\\
    &= \frac{0.032}{6.02 \times 10^{23}} \\
    &= 5.32 \times 10^{26} kg
  \end{align}
\]
(3)
\[
  \begin{align}
  \overline{v} &= 1.60 \times \sqrt{\frac{RT}{M}}\\
  &= 1.60 \times \sqrt{\frac{8.31J/mol \cdot K \times 300K}{0.032}}\\
  &= 446.58m/s
\end{align}
\]
(4)
\[
  \begin{align}
  \overline{\varepsilon _k} &= \frac{3}{2}kT\\
  &= \frac{3 \times 1.38 \times 10^{-23} J/K\times 300K}{2}\\
  &= 6.21 \times 10^{-21}J
  \end{align}
\]
\section{8.3
在容积为\(2.0l\)
的容器中,有内能为\(6.75\times10^2J\)
的氧气。(1)
求气体的压强; (2)若容器中分子总数为\(5.4 \times 10^{22}\)
个,求分子
的平均平动动能及气体的温度。
}

解:(1)\[
  \begin{align}
    E &= \frac{i}{2} \nu RT \\
    pV &= \nu RT\\
    p &= \frac{2E}{iV} \\
    &= \frac{2 \times 6.75 \times 10^2J}{5 \times 2.0 \times 10^{-3}m^3} \\
    &= 1.35 \times 10^{5} Pa
  \end{align} 
\]
(2)\[
  \begin{align}
    p &= \frac{2}{3}n \overline{\varepsilon_k} \\
    n &= \frac{N}{V}\\
    \overline{\varepsilon_k} &= \frac{3pV}{2N} \\
    &= \frac{3 \times 1.35 \times 10^5 Pa \times 2.0 \times 10^{-3} m^3 }{2 \times 5.4 \times 10^{22}}\\
    &= 7.5\times 10^{-21}J
  \end{align}
\]
(3)\[
  \begin{align}
    \overline{\varepsilon_k}&= \frac{3}{2}kT\\
    T &= \frac{2\overline{\varepsilon_k}}{3k}\\
    &= \frac{2 \times 7.5\times 10^{-21}}{3 \times 1.38 \times 10^{-23}J/k}\\
    &= 362.32K
  \end{align}
    \]

\section{1mol氢气,当温度为\(27^\circ C\)时,
求它的平均平动动能、平均转动动能及平动动能、转动动能、内能。}
解:
(1)
\[
  \begin{align}
  \overline{\varepsilon_{kr}} &= \frac{3}{2}kT\\
  &=\frac{3}{2} \times 1.38 \times 10^{-23} \times 300\\
  &=6.21 \times 10^{21}J
  \end{align}
\]
(2)
\[
  \begin{align}
    \overline{\varepsilon_{kt}} &= \frac{2}{2}kT\\
  &=\frac{2}{2} \times 1.38 \times 10^{-23} \times 300\\
  &=4.14 \times 10^{21}J
  \end{align}
\]

(3)\[
  \begin{align}
    E_{kt} &= \frac{3}{2} \nu RT\\
  &=\frac{3}{2} \times 8.31 \times 300\\
  &=3739.5J
  \end{align}
  \]
  (4)
  \[
  \begin{align}
    E_{kr} &= \frac{2}{2} \nu RT\\
  &=\frac{2}{2} \times 8.31 \times 300\\
  &=2493J
  \end{align}
  \]
  (5)
  \[
  \begin{align}
    E &= E_{kr} + E_{kt}\\
  &=3739.5 + 2493\\
  &=6232.5J
  \end{align}
  \]

\section{
  8.5速率分布函数\(f(v)\)的物理意义是什么? 试说明下列
  各量的物理意义(n为分子数密度,N为系统总分子数)。
}
解:\\ \(f(v)\)表示在温度为T的平衡状态下,速率在V附近单位速率区间 的分子数占总数的百分比。\\
\(f(v)dV\)表示速率落在\(v\)到\(v+dv\)区间内的分子数占总分子数的百分比。\\
\(nf(v)dv\)表示单位体积内速率落在v到v+dv区间内的分子数。\\
\(\int_{v1}^{v2} f(v) \,dv \)表示速率落在\(v_1\)到\(v_2\)区间内的分子数占总分子数的百分比。
\(\int_{\infty}^{0} f(v) \,dv \)表示在0到\(\infty\)速率区间内的分子数占总分子数的百分比。\\
\(\int_{v2}^{v1} Nf(v) \,dv\) 表示速率落在\(v_1\)到\(v_2\)区间内的分子数。
\section{
  8.6设有 N 个粒子的系统,速率分布如图所示。(1)说明曲线
  与横坐标所包围面积的含义; (2) 写出速率分布函数表达式
  (3)求\(a\)与\(v_0\)
  之间的关系;(4)求速率在\(1.5V_0 - 2.0V_0\)
  间隔内的粒
  子数;(5)求粒子的平均速率;(6)求\(0.5V_0-V_0\)
  区间内粒子的
  平均速率。
}
解:
(1)\\
(2)
\[
  f(v)=  \begin{cases}
    \frac{a}{Nv_0}v,\quad 0\leqslant v\leqslant v_0 \\
    \frac{a}{N},\quad v_0\leqslant v\leqslant 2v_0\\
    0, \quad v > 2v_0\\
    \end{cases}
\]
(3)
\[
  \begin{align}
    N &= \frac{1}{2} (v_0 + 2v_0)\cdot a\\
    a &= \frac{2N}{3V_0}
  \end{align}\]
(4)
\[
  \begin{align}
    \Delta N &= 0.5v_0 \cdot a\\
    &= 0.5v_0 \cdot \frac{2N}{3v_0}\\
    &= \frac{N}{3}
  \end{align}
  \]
(5)
\[
  \begin{align}
    \overline{v} &= \int_{v_0}^{0} v \cdot \frac{a}{Nv_0}v \,dv +\int_{2v_0}^{v_0} v \cdot \frac{a}{N}\,dv \\
    &= \frac{11}{6N}av^2_0\\
    &=\frac{11}{9}v_0
  \end{align}
  \]
\section{一真空管的真空度约为\(1.38\times 10^{-3} Pa\)
,试求在\(27^\circ C\)
时单位体积中的分子数及分子的平均自由程(设分子
的有效直径\(d = 3\times 10^{-10} m\)
)}

解:
\[
  \begin{align} 
    p &= nkT\\
    n &= \frac{p}{kT}\\
    &=\frac{1.38 \times 10^{-3}}{1.38 \times 10^{-23} \times 300}\\
    &= 3.33 \times 10^{17}m_{-3}\\
    \overline{\lambda } &= \frac{kT}{\sqrt{2} \times 3.14 \times (3 \times  10^{-10})^2 \times 3.33 \times 10^{17}}\\
    &= 7.5m\\
  \end{align}\]

\end{document}
